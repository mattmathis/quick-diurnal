\documentclass[sigconf]{acmart}

\usepackage{booktabs} % For formal tables


% Copyright
%\setcopyright{none}
%\setcopyright{acmcopyright}
%\setcopyright{acmlicensed}
\setcopyright{rightsretained}
%\setcopyright{usgov}
%\setcopyright{usgovmixed}
%\setcopyright{cagov}
%\setcopyright{cagovmixed}


% DOI
%\acmDOI{10.475/123_4}

% ISBN
%\acmISBN{123-4567-24-567/08/06}

%Conference
\acmConference[IMC'17]{Internet Measurement Conference}{November 2017}{London, UK}
\acmYear{2017}
\copyrightyear{2017}

%\acmPrice{15.00}


\begin{document}
\title{Diurnally Varying Network Performance is a Leading Signal of Network Congestion}
%\titlenote{Produces the permission block, and copyright information}
%\subtitle{Extended Abstract}
%\subtitlenote{The full version of the author's guide is available as \texttt{acmart.pdf} document}


\author{Peter Boothe}
%\orcid{1234-5678-9012}
\affiliation{%
  \institution{Google}
  \streetaddress{111 8th Ave}
  \city{New York}
  \state{NY}
  \postcode{10011}
}
\email{pboothe@google.com}

\author{Matt Mathis}
\affiliation{%
  \institution{Google}
  \streetaddress{1600 Amphitheatre Parkway}
  \city{Mountain View}
  \state{California}
  \postcode{94043}
}
\email{mattmathis@google.com}

\begin{abstract}
If network performance is measured to be cyclically varying on a 24-hour cycle,
then that network is likely to become congested soon. Thus, for
just-in-time network engineering, look for network segments with a
diurnally-varying performance.
\end{abstract}

%
% The code below was generated by the tool at
% http://dl.acm.org/ccs.cfm
%
\begin{CCSXML}
<ccs2012>
 <concept>
  <concept_id>10003033</concept_id>
  <concept_desc>Networks</concept_desc>
  <concept_significance>300</concept_significance>
 </concept>
 <concept>
  <concept_id>10003033.10003079.10011672</concept_id>
  <concept_desc>Networks~Network performance analysis</concept_desc>
  <concept_significance>500</concept_significance>
 </concept>
</ccs2012>  
\end{CCSXML}

%\ccsdesc[300]{Networks}
%\ccsdesc[500]{Networks~Network performance analysis}

% end generated code

\keywords{network performance, network measurement}

\maketitle

\section{Introduction}

\section{Diurnal Signals}
\subsection{Metrics of Interest}
\subsection{Methods of Calculating Diurnal Signal Strength}

\section{Experimental Validation}
\subsection{Presence of Diurnal Signals}
\subsection{Presence of Congestion}

\section{Conclusion}

\section{Bibliography}
\end{document}
